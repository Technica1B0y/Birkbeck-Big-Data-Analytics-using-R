\documentclass[]{article}
\usepackage{lmodern}
\usepackage{amssymb,amsmath}
\usepackage{ifxetex,ifluatex}
\usepackage{fixltx2e} % provides \textsubscript
\ifnum 0\ifxetex 1\fi\ifluatex 1\fi=0 % if pdftex
  \usepackage[T1]{fontenc}
  \usepackage[utf8]{inputenc}
\else % if luatex or xelatex
  \ifxetex
    \usepackage{mathspec}
  \else
    \usepackage{fontspec}
  \fi
  \defaultfontfeatures{Ligatures=TeX,Scale=MatchLowercase}
\fi
% use upquote if available, for straight quotes in verbatim environments
\IfFileExists{upquote.sty}{\usepackage{upquote}}{}
% use microtype if available
\IfFileExists{microtype.sty}{%
\usepackage{microtype}
\UseMicrotypeSet[protrusion]{basicmath} % disable protrusion for tt fonts
}{}
\usepackage[margin=1in]{geometry}
\usepackage{hyperref}
\hypersetup{unicode=true,
            pdftitle={CW1\_13154855\_A\_Guo},
            pdfauthor={Anyi Guo},
            pdfborder={0 0 0},
            breaklinks=true}
\urlstyle{same}  % don't use monospace font for urls
\usepackage{graphicx,grffile}
\makeatletter
\def\maxwidth{\ifdim\Gin@nat@width>\linewidth\linewidth\else\Gin@nat@width\fi}
\def\maxheight{\ifdim\Gin@nat@height>\textheight\textheight\else\Gin@nat@height\fi}
\makeatother
% Scale images if necessary, so that they will not overflow the page
% margins by default, and it is still possible to overwrite the defaults
% using explicit options in \includegraphics[width, height, ...]{}
\setkeys{Gin}{width=\maxwidth,height=\maxheight,keepaspectratio}
\IfFileExists{parskip.sty}{%
\usepackage{parskip}
}{% else
\setlength{\parindent}{0pt}
\setlength{\parskip}{6pt plus 2pt minus 1pt}
}
\setlength{\emergencystretch}{3em}  % prevent overfull lines
\providecommand{\tightlist}{%
  \setlength{\itemsep}{0pt}\setlength{\parskip}{0pt}}
\setcounter{secnumdepth}{0}
% Redefines (sub)paragraphs to behave more like sections
\ifx\paragraph\undefined\else
\let\oldparagraph\paragraph
\renewcommand{\paragraph}[1]{\oldparagraph{#1}\mbox{}}
\fi
\ifx\subparagraph\undefined\else
\let\oldsubparagraph\subparagraph
\renewcommand{\subparagraph}[1]{\oldsubparagraph{#1}\mbox{}}
\fi

%%% Use protect on footnotes to avoid problems with footnotes in titles
\let\rmarkdownfootnote\footnote%
\def\footnote{\protect\rmarkdownfootnote}

%%% Change title format to be more compact
\usepackage{titling}

% Create subtitle command for use in maketitle
\newcommand{\subtitle}[1]{
  \posttitle{
    \begin{center}\large#1\end{center}
    }
}

\setlength{\droptitle}{-2em}

  \title{CW1\_13154855\_A\_Guo}
    \pretitle{\vspace{\droptitle}\centering\huge}
  \posttitle{\par}
    \author{Anyi Guo}
    \preauthor{\centering\large\emph}
  \postauthor{\par}
      \predate{\centering\large\emph}
  \postdate{\par}
    \date{16/10/2018}


\begin{document}
\maketitle

\begin{enumerate}
\def\labelenumi{\arabic{enumi}.}
\tightlist
\item
  Statistical learning methods
\end{enumerate}

For each of parts (a) through (d), indicate whether we would generally
expect the performance of a flexible statistical learning method to be
better or worse than an inflexible method. Justify your answer.

\begin{enumerate}
\def\labelenumi{(\alph{enumi})}
\tightlist
\item
  The sample size n is extremely large, and the number of predictors p
  is small.\\
\item
  The number of predictors p is extremely large, and the number of
  observations n is small.\\
\item
  The relationship between the predictors and response is highly
  non-linear.\\
\item
  The variance of the error terms, i.e.~σ2 = Var(ε), is extremely high.
\end{enumerate}

\begin{enumerate}
\def\labelenumi{\arabic{enumi}.}
\setcounter{enumi}{1}
\tightlist
\item
  Descriptive analysis
\end{enumerate}

In a higher educational institution the comprehensive applied
mathematics exam is comprised of two parts. On the first day, 20
students took the exam, the results of which are presented below:

Oral exam results: 4, 1, 4, 5, 3, 2, 3, 4, 3, 5, 2, 2, 4, 3, 5, 5, 1, 1,
1, 2. Writtenexamresults: 2,3,1,4,2,5,3,1,2,1,2,2,1,1,2,3,1,2,3,4.

\begin{enumerate}
\def\labelenumi{(\alph{enumi})}
\tightlist
\item
  Use R to calculate the mean, the mode, the median, the variance and
  the standard deviation of the oral and written exams separately and
  together as well.\\
\item
  Find the covariance and correlation between the oral and written exam
  scores.\\
\item
  Is there a positive or negative or no correlation between the two?\\
\item
  Is there causation between the two? Justify your answers.\\
\end{enumerate}

\begin{enumerate}
\def\labelenumi{\arabic{enumi}.}
\setcounter{enumi}{2}
\tightlist
\item
  Descriptive analysis (10\% \textbar{} 0\%) This exercise involves the
  Auto data set studied in the class. Make sure that the missing values
  have been removed from the data.
\end{enumerate}

\begin{enumerate}
\def\labelenumi{(\alph{enumi})}
\tightlist
\item
  Which of the predictors are quantitative, and which are qualitative?
\item
  What is the range of each quantitative predictor? You can answer this
  using the range() function.
\item
  What is the mean and standard deviation of each quantitative
  predictor?
\item
  Now remove the 10th through 85th observations. What is the range,
  mean, and standard deviation of each predictor in the subset of the
  data that remains? 1
\item
  Using the full data set, investigate the predictors graphically, using
  scatterplots or other tools of your choice. Create some plots
  highlighting the relationships among the predictors. Comment on your
  findings.
\item
  Suppose that we wish to predict gas mileage (mpg) on the basis of the
  other variables. Do your plots suggest that any of the other variables
  might be useful in predicting mpg? Justify your answer.
\end{enumerate}

\begin{enumerate}
\def\labelenumi{\arabic{enumi}.}
\setcounter{enumi}{3}
\tightlist
\item
  Linear regression (20\% \textbar{} 20\%) This question involves the
  use of simple linear regression on the Auto data set.
\end{enumerate}

\begin{enumerate}
\def\labelenumi{(\alph{enumi})}
\tightlist
\item
  Use the lm() function to perform a simple linear regression with mpg
  as the response and horsepower as the predictor. Use the summary()
  function to print the results. Comment on the output. For example:
\end{enumerate}

\begin{enumerate}
\def\labelenumi{\roman{enumi}.}
\tightlist
\item
  Is there a relationship between the predictor and the response?
\item
  How strong is the relationship between the predictor and the response?
\item
  Is the relationship between the predictor and the response positive or
  negative?
\item
  What is the predicted mpg associated with a horsepower of 98? What are
  the associated 95\% confidence and prediction intervals?
\end{enumerate}

\begin{enumerate}
\def\labelenumi{(\alph{enumi})}
\setcounter{enumi}{1}
\tightlist
\item
  Plot the response and the predictor. Use the abline() function to
  display the least squares regression line.
\item
  Plot the 95\% confidence interval and prediction interval in the same
  plot as (b) using different colours and legends.
\end{enumerate}

\begin{enumerate}
\def\labelenumi{\arabic{enumi}.}
\setcounter{enumi}{4}
\tightlist
\item
  Logistic regression (10\% \textbar{} 20\%) Using the Boston data set,
  fit classification models in order to predict whether a given suburb
  has a crime rate above or below the median. Explore logistic
  regression models using various subsets of the predictors. Describe
  your findings.
\item
  Resampling methods (20\% \textbar{} 0\%) Suppose that we use some
  statistical learning method to make a prediction for the response Y
  for a particular value of the predictor X. Carefully describe how we
  might estimate the standard deviation of our prediction.
\item
  Resampling methods We will now perform cross-validation on a simulated
  data set. (a) Generate a simulated data set as follows: set.seed(500)
  y = rnorm(500) x = 4 - rnorm(500) y = x - 2\emph{x\^{}2 + 3}x\^{}4 +
  rnorm(500) (20\% \textbar{} 20\%) In this data set, what is n and what
  is p? Write out the model used to generate the data in equation form.
\end{enumerate}

\begin{enumerate}
\def\labelenumi{(\alph{enumi})}
\setcounter{enumi}{1}
\tightlist
\item
  Create a scatterplot of X against Y. Comment on what you find.
\item
  Set the seed to be 23, and then compute the LOOCV and 10-fold CV
  errors that result from fitting the following four models using least
  squares:
\end{enumerate}

\begin{enumerate}
\def\labelenumi{\roman{enumi}.}
\item
  Y = β0 + β1X + ε
\item
  Y =β0 +β1X+β2X2 +ε 2
\item
  Y =β0 +β1X+β2X2 +β3X3 +ε
\item
  Y =β0 +β1X+β2X2 +β3X3 +β4X4 +ε. Note you may find it helpful to use
  the data.frame() function to create a single data set containing both
  X and Y.
\end{enumerate}

\begin{enumerate}
\def\labelenumi{(\alph{enumi})}
\setcounter{enumi}{3}
\tightlist
\item
  Repeat (c) using random seed 46, and report your results. Are your
  results the same as what you got in (c)? Why?
\item
  Which of the models in (c) had the smallest LOOCV and 10-fold CV
  error? Is this what you expected? Explain your answer.
\item
  Comment on the statistical significance of the coefficient estimates
  that results from fitting each of the models in (c) using least
  squares. Do these results agree with the conclusions drawn based on
  the cross-validation results?
\end{enumerate}


\end{document}
